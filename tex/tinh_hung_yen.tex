\documentclass{article}
\usepackage[vietnamese]{babel}
\usepackage[letterpaper,top=1cm,bottom=1cm,left=1.5cm,right=1.5cm,marginparwidth=1.75cm]{geometry}
\usepackage{amsmath}
\usepackage{graphicx}
\usepackage[colorlinks=true, allcolors=blue]{hyperref}
\title{Cẩm nang du lịch Tỉnh Hưng Yên}

\begin{document}
\begin{center}
    \fontsize{18}{20}\textbf{Cẩm nang du lịch Tỉnh Hưng Yên}
\end{center}
\begin{abstract}
    (Giới thiệu)
\end{abstract}
\section*{Điểm du lịch}

\section{Grand World – Vũ trụ giải trí bất tận}
\begin{itemize}
    \item{\textbf{Địa chỉ}} Xã Nghĩa Trụ, Văn Giang, Hưng Yên

    \item{\textbf{Giờ mở cửa:}} Cả ngày

    \item{\textbf{Giá vé/ chi phí:}} Miễn phí

    \item{\textbf{Phân loại điểm du lịch}}

    \item{\textbf{Nguồn thông tin}}

    \item{\textbf{Thông tin về địa điểm:}} Khung cảnh châu Âu cổ điển tại Grand World mang đến cho du khách trải nghiệm như đang lạc vào thành phố Venice thơ mộng. Dãy phố với những ngôi nhà rực rỡ sắc màu và dòng kênh đào yên ả tạo nên không khí lãng mạn. Cuối dòng Venice là cầu đi bộ Đông Tây nối liền với phân khu K-Town, nơi du khách có thể tận hưởng không khí sôi động của khu mua sắm và chiêm ngưỡng văn hoá Hàn Quốc cực “swag”. Grand World còn là thiên đường cho những ai yêu thích chụp ảnh check-in. Mỗi góc phố, cung đường đều được thiết kế cho các du khách tha hồ sống ảo, ghi lại những khoảnh khắc tuyệt vời. Những địa điểm nổi bật có thể kể đến như là tháp đồng hồ cổ kính, quảng trường Grande, đài phun nước Amore… Grand World không chỉ là nơi giải trí mà còn là thiên đường ẩm thực và mua sắm. Phố mua sắm tại Grand World hứa hẹn mang đến trải nghiệm mua sắm tuyệt vời nhất với các thương hiệu từ cao cấp đến local brand như Adidas, BOO, F1 Enter, Áo dài Sài Gòn Hà Anh… Ngoài ra, nơi đây còn được coi là “thiên đường ăn uống” với đa dạng các món ngon từ đường phố, ẩm thực truyền thống tới đồ Âu như Nét Huế, TexGrill, Bò Tơ Quán Mộc… Các thương hiệu cà phê nổi tiếng như Trung Nguyên, Katinat, Highlands cũng đã có mặt tại Grand World, tạo nên trải nghiệm mĩ mãn dành cho các du khách. Khi màn đêm buông xuống, Grand World trở nên tráng lệ và xa hoa với vở diễn thực cảnh The Grand Voyage. Với công nghệ mapping 3D, show diễn đã tái hiện một hải trình giao thương từ châu Âu, đi qua con đường tơ lụa và khám phá vùng đất phía biển Đông với thương cảng Hội An – Việt Nam trong truyền thuyết.
\end{itemize}

\begin{itemize}
    \item{\textbf{Các thông tin khác - Cách di chuyển}}
\end{itemize}

\section{Đền Chử Đồng Tử – Một trong tứ bất tử của Việt Nam}
\begin{itemize}
    \item{\textbf{Địa chỉ}} Bình Minh, Khoái Châu, Hưng Yên

    \item{\textbf{Giờ mở cửa:}} Cả ngày

    \item{\textbf{Giá vé/ chi phí:}} Miễn phí

    \item{\textbf{Phân loại điểm du lịch}}

    \item{\textbf{Nguồn thông tin}}

    \item{\textbf{Thông tin về địa điểm:}} Đền Chử Đồng Tử là một công trình lịch sử nổi tiếng tại vùng trung du Bắc Bộ, đặc biệt là ở Khoái Châu, Hưng Yên. Hai ngôi đền nổi tiếng nhất là đền Đa Hoà và đền Dạ Trạch thờ Đức thánh Chử Đồng Tử cùng công chúa Tiên Dung và công chúa Tây Sa, được xếp hạng là di tích lịch sử cấp quốc gia. Kiến trúc của đền bao gồm 18 ngôi nhà mái ngói cong hình 18 con rồng cách điệu, tái hiện hình ảnh 18 con thuyền của công chúa Tiên Dung khi du ngoạn trên sông. Đền Chử Đồng Tử – Tiên Dung có nhiều cổ vật quý hiếm, bao gồm lục bình Bách thọ – cỗ ngai cổ nhất còn tồn tại ở Việt Nam. Các kiệu hát, tượng chạm và đúc trổ tinh tế cũng là nét nổi bật của Đền. Khám phá Đền, du khách sẽ bị mê hoặc bởi không khí yên bình và thanh tịnh.

\end{itemize}

\begin{itemize}
    \item{\textbf{Các thông tin khác - Cách di chuyển}}
\end{itemize}

\section{Chùa Phúc Lâm – “Chùa Vàng” của Việt Nam}
\begin{itemize}
    \item{\textbf{Địa chỉ}} Phù Ủng, Ân Thi, Hưng Yên

    \item{\textbf{Giờ mở cửa:}} Cả ngày

    \item{\textbf{Giá vé/ chi phí:}} Miễn phí

    \item{\textbf{Phân loại điểm du lịch}}

    \item{\textbf{Nguồn thông tin}}

    \item{\textbf{Thông tin về địa điểm:}} Chùa Phúc Lâm – Một ngôi “chùa Vàng” của Việt Nam là địa điểm tâm linh lâu đời và đã thu hút nhiều du khách với vẻ ngoài được dát vàng rực rỡ. Chùa Phúc Lâm bao gồm toà Tiền đường và toà Thượng Điện cùng với 4 tòa tháp, mỗi tháp được trang trí với nhiều pho tượng Phật lớn, tạo nên vẻ đẹp thiêng liêng và huyền bí và thanh tịnh. Lối kiến trúc độc đáo và nghệ thuật trạm trổ tinh tế của ngôi chùa cổ hơn 100 năm tuổi này đã tạo nên một không gian thanh tịnh, thu hút nhiều người đến cầu tài, cầu phúc. Sau khi vãn cảnh chùa, du khách có thể thưởng thức ẩm thực đặc sản Hưng Yên như bánh tẻ, bún cá, tương Bần…
\end{itemize}

\begin{itemize}
    \item{\textbf{Các thông tin khác - Cách di chuyển}}
\end{itemize}

\section{Làng hương Cao Thôn – Lưu giữ nét đẹp tín ngưỡng}
\begin{itemize}
    \item{\textbf{Địa chỉ}} Bảo Khê, Hưng Yên

    \item{\textbf{Giờ mở cửa:}} Cả ngày

    \item{\textbf{Giá vé/ chi phí:}} Miễn phí

    \item{\textbf{Phân loại điểm du lịch}}

    \item{\textbf{Nguồn thông tin}}

    \item{\textbf{Thông tin về địa điểm:}} Nổi tiếng với loại hương thơm nhẹ, lưu giữ mùi lâu, làng hương Cao Thôn là địa điểm du lịch ở Hưng Yên phù hợp với những ai yêu thích tín ngưỡng truyền thống. Mỗi cây hương tại đây đều được chế tác một cách tỉ mỉ và công phu bởi các nghệ nhân tại làng. Ngoài các sản phẩm truyền thống như cây hương, làng hương Thôn Cao còn nổi tiếng với nhiều loại hương khác như hương nén, hương sào, hương vòng… Hương tại Cao Thôn với mùi hương thanh nhẹ. Với hơn trăm năm lịch sử, làng hương vẫn giữ vững danh tiếng và thu hút nhiều sự quan tâm của thị trường nội địa và quốc tế như Ấn Độ, Trung Quốc… Làng hương Thôn Cao là biểu tượng của sự nghiên cứu và bảo tồn nghệ thuật chế tác hương truyền thống của Việt Nam.
\end{itemize}

\begin{itemize}
    \item{\textbf{Các thông tin khác - Cách di chuyển}}
\end{itemize}

\section{Phố Hiến – Thứ nhất Kinh Kỳ, thứ nhì Phố Hiến}
\begin{itemize}
    \item{\textbf{Địa chỉ}} Trung tâm thành phố Hưng Yên, tỉnh Hưng Yên

    \item{\textbf{Giờ mở cửa:}} Cả ngày

    \item{\textbf{Giá vé/ chi phí:}} Miễn phí

    \item{\textbf{Phân loại điểm du lịch}}

    \item{\textbf{Nguồn thông tin}}

    \item{\textbf{Thông tin về địa điểm:}} Khu di tích quốc gia đặc biệt – Phố Hiến nằm trong lòng thành phố Hưng Yên, là một điểm đến tuyệt vời cho những ai yêu thích khám phá văn hoá và lịch sử Việt Nam. Nơi đây là biểu tượng cho sự đa dạng và hòa quyện giữa nền văn hoá đặc trưng của Việt Nam, Trung Hoa và phong cách kiến trúc phương Tây. Quần thể di tích Phố Hiến bao gồm 16 di tích nổi bật, tiêu biểu có thể kể đến như chùa Chuông, Văn Miếu Xích Đằng, Đền Mẫu… Bên cạnh những di tích lịch sử, Phố Hiến còn nổi tiếng với ẩm thực đa dạng và phong phú như nhãn lồng, bún thang lươn, chè sen… Khám phá Phố Hiến mang lại trải nghiệm tuyệt vời với không khí trong lành, đường phố yên bình với những di tích lịch sử. Bạn có thể thuê một chiếc xe đạp hoặc xe máy để dễ dàng di chuyển và khám phá mọi ngóc ngách của nơi đây.
\end{itemize}

\begin{itemize}
    \item{\textbf{Các thông tin khác - Cách di chuyển}}
\end{itemize}

\section{Làng Nôm – Không gian lắng đọng giữa thời hiện đại}
\begin{itemize}
    \item{\textbf{Địa chỉ}} Đại Đồng, Văn Lâm, Hưng Yên

    \item{\textbf{Giờ mở cửa:}} Cả ngày

    \item{\textbf{Giá vé/ chi phí:}} Miễn phí

    \item{\textbf{Phân loại điểm du lịch}}

    \item{\textbf{Nguồn thông tin}}

    \item{\textbf{Thông tin về địa điểm:}} Làng Nôm, một địa điểm du lịch ở Hưng Yên, là nơi có sự kết hợp hài hoà giữa những chứng tích lịch sử, kiến trúc truyền thống và phong cảnh hữu tình. Những ngôi nhà cổ được xây dựng theo kiến trúc với vườn tược, sân nhà… tạo nên một không gian truyền thống với nét đẹp nhẹ nhàng. Điểm đặc biệt của làng Nôm là ba giếng cổ có tuổi đời hàng nghìn năm. Ba giếng cổ không chỉ nổi bật với những hàng gạch và phiến đá cổ nguyên khối mà còn là nguồn nước tinh khiết, quý giá cho cả làng. Ngoài ra, một địa điểm mà bạn nhất định không được bỏ qua khi đến với nơi đây là chùa Nôm. Chùa Nôm, hay còn gọi là Linh Thông Cổ Tự, là nơi linh thiêng với hơn 100 bức tượng cổ được làm từ đất nung. Những bức tượng này là những tác phẩm nghệ thuật có giá trị lịch sử và văn hoá dân tộc, thể hiện sự khéo léo và tinh tế của người nghệ nhân xưa.
\end{itemize}

\begin{itemize}
    \item{\textbf{Các thông tin khác - Cách di chuyển}}
\end{itemize}

\section{Làng Thủ Sỹ – Làng nghề đan đó hơn 200 năm tuổi}
\begin{itemize}
    \item{\textbf{Địa chỉ}} Thủ Sỹ, Tiên Lữ, Hưng Yên

    \item{\textbf{Giờ mở cửa:}} Cả ngày

    \item{\textbf{Giá vé/ chi phí:}} Miễn phí

    \item{\textbf{Phân loại điểm du lịch}}

    \item{\textbf{Nguồn thông tin}}

    \item{\textbf{Thông tin về địa điểm:}} Làng nghề Thủ Sỹ nổi tiếng với hình ảnh đậm chất truyền thống làng quê Bắc Bộ với những ngôi nhà mái ngói đỏ, nếp nhà ba gian cổ kính và những luỹ tre xanh. Đến với Thủ Sỹ, du khách có cơ hội tận hưởng cuộc sống quê hương bình yên và giản dị của miền quê Việt Nam. Đặc biệt, làng Thủ Sỹ còn rất nổi tiếng với nghề đan đó truyền thống. Ghé thăm các gia đình trong làng, bạn sẽ bắt gặp hình ảnh các cụ ông, cụ bà ngồi trước hiên, bàn tay thoăn thoắt đang mải mê đan đó. Không gian Thủ Sỹ được bao trùm bởi tiếng chẻ tre, chẻ nứa… tạo nên bức tranh rộn ràng, sống động của làng quê.
\end{itemize}

\begin{itemize}
    \item{\textbf{Các thông tin khác - Cách di chuyển}}
\end{itemize}

\section{Cánh đồng hoa Cúc Chi – Rực rỡ một góc trời}
\begin{itemize}
    \item{\textbf{Địa chỉ}} Nghĩa Trai, Văn Lâm, Hưng Yên

    \item{\textbf{Giờ mở cửa:}} Cả ngày

    \item{\textbf{Giá vé/ chi phí:}} 20.000 VNĐ

    \item{\textbf{Phân loại điểm du lịch}}

    \item{\textbf{Nguồn thông tin}}

    \item{\textbf{Thông tin về địa điểm:}} Cánh đồng hoa cúc chi của Nghĩa Trai, Hưng Yên là một bức tranh thiên nhiên tuyệt vời, hút hồn mỗi du khách ghé thăm tại đây. Hoa cúc chi, hay còn gọi là hoa kim cúc, hoàng cúc, là biểu tượng của Phố Hiến khi cái lạnh mùa đông về. Chỉ mất khoảng một giờ lái xe máy từ trung tâm Hà Nội, bạn sẽ đến được với cánh đồng hoa cúc chi rực rỡ. Cánh đồng hoa cúc chi Hưng Yên vào những ngày cuối năm với màu vàng tươi rực rỡ làm sáng bừng cả một vùng quê yên bình và thơ mộng. Chỉ cần diện một bộ trang phục xinh xắn, nhẹ nhàng tại đây cũng đủ để cho bạn tha hồ check-in sống ảo rồi đó!
\end{itemize}

\begin{itemize}
    \item{\textbf{Các thông tin khác - Cách di chuyển}}
\end{itemize}

\section{Làng nghề Tương Bần Yên Nhân – Nơi gìn giữ tinh hoa ẩm thực}
\begin{itemize}
    \item{\textbf{Địa chỉ}} Thị xã Mỹ Hào, Hưng Yên

    \item{\textbf{Giờ mở cửa:}} Cả ngày

    \item{\textbf{Giá vé/ chi phí:}} Miễn phí

    \item{\textbf{Phân loại điểm du lịch}}

    \item{\textbf{Nguồn thông tin}}

    \item{\textbf{Thông tin về địa điểm:}} Làng nghề tương Bần Yên Nhân với truyền thống làm nghề tương tồn tại hàng trăm năm được bảo tồn từ đời này sang đời khác. Ngay khi bước vào làng, du khách sẽ bắt gặp mùi thơm đặc trưng của tương. Mỗi chai tương ở đây đều là sản phẩm tự chế biến của các gia đình trong làng, đảm bảo chất lượng, an toàn, vệ sinh. Nghề làm tương ở đây có lịch sử lâu dài và nổi tiếng với hương vị đậm đà, thơm ngon. Quy trình nấu tương có vẻ đơn giản với những nguyên liệu tự nhiên như ngô, gạo… nhưng đòi hỏi sự khéo léo, tỉ mỉ, tạo nên những sản phẩm tương Bần ngon hơn so với các vùng khác. Du khách ghé thăm có thể dễ dàng mua tương về làm quà, dùng làm nước chấm cho mâm cơm thêm ngon miệng.
\end{itemize}

\begin{itemize}
    \item{\textbf{Các thông tin khác - Cách di chuyển}}
\end{itemize}

\section{Chùa Hương Lãng – Chứng nhân lịch sử nghệ thuật thời Lý}
\begin{itemize}
    \item{\textbf{Địa chỉ}} Thôn Chùa, Văn Lâm, Hưng Yên

    \item{\textbf{Giờ mở cửa:}} Cả ngày

    \item{\textbf{Giá vé/ chi phí:}} Miễn phí

    \item{\textbf{Phân loại điểm du lịch}}

    \item{\textbf{Nguồn thông tin}}

    \item{\textbf{Thông tin về địa điểm:}} Xuyên suốt lịch sử và thời gian, chùa Hương Lãng đã trải qua nhiều lần tu sửa và xây mới. Ngày nay, ngôi chùa được xây dựng nhỏ hơn so với nền móng cũ, bao gồm nhà đại bái, tiền đường và hậu cung. Chùa Hương Lãng còn là nơi lưu giữ 2 di vật quý giá được công nhận là Bảo vật quốc gia là hệ thống thành bậc đá và bệ tượng sư tử đá. Đây chính là những chứng nhân lịch sử về sự phồn thịnh và phát triển của nghệ thuật tại thời nhà Lý. Ngoài ra, chùa Hương Lãng còn lưu giữ nhiều hiện vật quý từ thời nhà Lý như 4 cột đá vuông góc đỡ các xà đá của công trình và nhiều tảng đá chân cột chạm khắc cánh hoa sen… Với những giá trị về văn hoá, lịch sử, nghệ thuật và kiến trúc, chùa Hương Lãng đã được xếp hạng là Di tích kiến trúc nghệ thuật cấp quốc gia năm 1974.
\end{itemize}

\begin{itemize}
    \item{\textbf{Các thông tin khác - Cách di chuyển}}
\end{itemize}

\section{Đền Chử Đồng Tử}
\begin{itemize}
    \item{\textbf{Địa chỉ}} Xã Bình Minh, huyện Khoái Châu, tỉnh Hưng Yên

    \item{\textbf{Giờ mở cửa:}} Cả ngày

    \item{\textbf{Giá vé/ chi phí:}} Miễn phí

    \item{\textbf{Phân loại điểm du lịch}} 

    \item{\textbf{Nguồn thông tin}} 

    \item{\textbf{Thông tin về địa điểm:}} Đền Chử Đồng Tử là điểm du lịch tâm linh nổi tiếng nhất tại Hưng Yên, đã được xếp hạng di tích lịch sử cấp quốc gia. Đền Chử Đồng Tử bao gồm hai ngôi đền đó là đền Đa Hòa và đền Dạ Trạch, nơi đây thờ Đức thánh Chử Đồng Tử, công chúa Tiên Dung và công chúa Tây Sa. Đền Chử Đồng Tử là một trong những địa điểm du lịch Hưng Yên thu hút nhiều du khách. Đền Chử Đồng sở hữu kiến trúc độc đáo với 18 ngôi nhà mái cong hình 18 con rồng thiết kế cách điệu, tái hiện hình ảnh 18 con thuyền mà khi xưa công chúa Tiên Dung du ngoạn trên sông. Trong đền lưu giữ nhiều cổ vật quý hiếm như lục bình Bách thọ - cỗ ngai lâu đời nhất còn tồn tại ở Việt Nam, bên cạnh đó còn có các kiệu hát, tượng chạm trổ tinh tế. Lễ hội Chử Đồng Tử diễn ra từ ngày 10 - 12 tháng 2 Âm lịch hàng năm. Lễ hội này bao gồm rất nhiều hoạt động như lễ rước nước, chọi gà, múa rồng, ca hát trên thuyền, đập niêu… Ngoài ra, người dân bản địa còn thực hiện nhiều nghi thức truyền thống để cầu mong mưa thuận gió hòa, mùa màng bội thu. Múa rồng là một hoạt động nổi bật trong lễ hội Chử Đồng Tử.
\end{itemize}

\begin{itemize}
    \item{\textbf{Các thông tin khác - Cách di chuyển}} 
\end{itemize}

\section{Phố Hiến}
\begin{itemize}
    \item{\textbf{Địa chỉ}} Thành phố Hưng Yên

    \item{\textbf{Giờ mở cửa:}} Cả ngày

    \item{\textbf{Giá vé/ chi phí:}} Miễn phí

    \item{\textbf{Phân loại điểm du lịch}} 

    \item{\textbf{Nguồn thông tin}} 

    \item{\textbf{Thông tin về địa điểm:}} Phố Hiến tọa lạc tại trung tâm thành phố Hưng Yên, đây là khu di tích quốc gia đặc biệt, thu hút những ai yêu thích khám phá lịch sử và văn hóa Việt Nam. Phố Hiến là biểu tượng của sự đa dạng, hòa quyện giữa các nền văn hóa Việt Nam, Trung Hoa và cả phong cách kiến trúc phương Tây. Đến Phố Hiến, bạn sẽ được ghé thăm những di tích tiêu biểu như Chùa Chuông, Đền Mẫu, Văn Miếu Xích Đằng…, được thưởng thức những món ăn ngon như nhãn lồng, chè sen, bún thang lươn… Để thoải mái di chuyển đến mọi ngóc ngách tại Phố Hiến, bạn có thể thuê xe đạp hoặc xe máy tự lái. Phố Hiến là khu di tích quốc gia đặc biệt, vẫn còn nguyên giá trị văn hóa to lớn cho đến ngày nay.
\end{itemize}

\begin{itemize}
    \item{\textbf{Các thông tin khác - Cách di chuyển}} 
\end{itemize}

\section{Chùa Hương Lãng}
\begin{itemize}
    \item{\textbf{Địa chỉ}} Thôn Chùa, xã Minh Hải, huyện Văn Lâm, tỉnh Hưng Yên

    \item{\textbf{Giờ mở cửa:}} Cả ngày

    \item{\textbf{Giá vé/ chi phí:}} Miễn phí

    \item{\textbf{Phân loại điểm du lịch}} 

    \item{\textbf{Nguồn thông tin}} 

    \item{\textbf{Thông tin về địa điểm:}} Theo sử sách chép lại, chùa Hương Lãng được xây dựng vào năm 1115 dưới thời nhà Lý. Trải qua nhiều lần tu sửa và xây mới, ngôi chùa hiện tại nhỏ hơn so với nền móng cũ, bao gồm nhà đại bái, tiền đường, hậu cung. Năm 1974, chùa Hương Lãng được xếp hạng Di tích kiến trúc nghệ thuật cấp quốc gia. Đến chùa Hương Lãng, bạn sẽ được chiêm ngưỡng những hiện vật quý giá như tượng sư tử đá, hệ thống thành bậc đá. Đây vốn là hai di vật được công nhận là Bảo vật quốc gia. Bên cạnh đó còn có những hiện vật quý từ thời nhà Lý như bốn cột đá vuông góc đỡ các xà đá, những tảng đá chân cột chạm khắc hoa cúc, hoa sen. Khi đi du lịch Hưng Yên, bạn hãy ghé thăm chùa Hương Lãng để chiêm ngưỡng những hiện vật quý giá.
\end{itemize}

\begin{itemize}
    \item{\textbf{Các thông tin khác - Cách di chuyển}} 
\end{itemize}

\section{Chùa Phúc Lâm}
\begin{itemize}
    \item{\textbf{Địa chỉ}} Xã Phù Ủng, huyện Ân Thi, tỉnh Hưng Yên

    \item{\textbf{Giờ mở cửa:}} Cả ngày

    \item{\textbf{Giá vé/ chi phí:}} Miễn phí

    \item{\textbf{Phân loại điểm du lịch}} 

    \item{\textbf{Nguồn thông tin}} 

    \item{\textbf{Thông tin về địa điểm:}} Chùa Phúc Lâm là một trong những điểm du lịch tâm linh lâu đời, thu hút rất nhiều du khách đến dâng hương và cầu tài, cầu phúc. Chùa Phúc Lâm bao gồm hai tòa chính là tòa Tiền đường và tòa Thượng điện, bên cạnh đó còn có bốn tòa tháp được trang trí công phu, tạo nên vẻ đẹp thiêng liêng và huyền bí. Đến Chùa Phúc Lâm, bạn sẽ được chiêm ngưỡng kiến trúc độc đáo cùng nghệ thuật chạm trổ tinh tế của ngôi chùa cổ hơn 100 năm tuổi. Đặc biệt, ngôi chùa được dát vàng rực rỡ, tạo nên vẻ ngoài nổi bật và lộng lẫy giữa khung cảnh thiên nhiên yên bình.
\end{itemize}

\begin{itemize}
    \item{\textbf{Các thông tin khác - Cách di chuyển}} 
\end{itemize}

\section{Làng hương Cao Thôn}
\begin{itemize}
    \item{\textbf{Địa chỉ}} Xã Bảo Khê, thành phố Hưng Yên

    \item{\textbf{Giờ mở cửa:}} Cả ngày

    \item{\textbf{Giá vé/ chi phí:}} Miễn phí

    \item{\textbf{Phân loại điểm du lịch}} 

    \item{\textbf{Nguồn thông tin}} 

    \item{\textbf{Thông tin về địa điểm:}} Làng hương Cao Thôn là địa điểm du lịch Hưng Yên phù hợp với những ai yêu thích tín ngưỡng truyền thống. Tại đây, mỗi cây hương đều được các nghệ nhân địa phương chế tác một cách tỉ mỉ và công phu, mang đến mùi hương thơm nhẹ, lưu giữ mùi lâu. Trải qua hàng trăm năm tồn tại và phát triển, làng hương này vẫn giữ vững danh tiếng, được ưa chuộng ở thị trường nội địa và các nước láng giềng. Giờ đây, làng hương Cao Thôn không chỉ dừng lại ở làng nghề truyền thống mà đã trở thành biểu tượng của nghệ thuật chế tác hương, qua đó lưu giữ nét đẹp tín ngưỡng của dân tộc Việt Nam.
\end{itemize}

\begin{itemize}
    \item{\textbf{Các thông tin khác - Cách di chuyển}} 
\end{itemize}

\section{Làng Nôm}
\begin{itemize}
    \item{\textbf{Địa chỉ}} Xã Đại Đồng, huyện Văn Lâm, tỉnh Hưng Yên

    \item{\textbf{Giờ mở cửa:}} Cả ngày

    \item{\textbf{Giá vé/ chi phí:}} Miễn phí

    \item{\textbf{Phân loại điểm du lịch}} 

    \item{\textbf{Nguồn thông tin}} 

    \item{\textbf{Thông tin về địa điểm:}} Làng Nôm là một trong những điểm đến thú vị bạn nên ghé thăm khi đi du lịch Hưng Yên. Đến đây, bạn sẽ được ngắm nhìn những ngôi nhà cổ in đậm dấu ấn của kiến trúc truyền thống, được chiêm ngưỡng những chứng tích lịch sử và đắm chìm trong phong cảnh thiên nhiên hữu tình. Đặc biệt, Làng Nôm có ba giếng cổ tuổi đời hàng nghìn năm. Ba giếng cổ vừa nổi bật với những hàng gạch, phiến đá cổ nguyên khối vừa mang đến nguồn nước tinh khiết cho cả làng. Ngoài ra, bạn còn có thể ghé thăm chùa Nôm - ngôi chùa linh thiêng với hơn 100 bức tượng cổ làm từ đất nung. Đến làng Nôm Hưng Yên, bạn sẽ được chiêm ngưỡng những ngôi nhà cổ mang đậm dấu ấn của kiến trúc truyền thống.
\end{itemize}

\begin{itemize}
    \item{\textbf{Các thông tin khác - Cách di chuyển}} 
\end{itemize}

\section{Làng Thủ Sỹ}
\begin{itemize}
    \item{\textbf{Địa chỉ}} Xã Thủ Sỹ, huyện Tiên Lữ, tỉnh Hưng Yên

    \item{\textbf{Giờ mở cửa:}} Cả ngày

    \item{\textbf{Giá vé/ chi phí:}} Miễn phí

    \item{\textbf{Phân loại điểm du lịch}} 

    \item{\textbf{Nguồn thông tin}} 

    \item{\textbf{Thông tin về địa điểm:}} Làng nghề Thủ Sỹ nổi tiếng với nghề đan đó truyền thống. Đến đây, bạn sẽ dễ dàng bắt gặp hình ảnh những cụ ông, cụ bà với bàn tay thoăn thoắt đang ngồi trước hiên đan đó. Không gian trong làng được bao trùm bởi những tiếng chẻ tre, chẻ nứa đầy sống động, tạo nên một khung cảnh đẹp bình dị như tranh vẽ. Khi du lịch Hưng Yên, ghé làng nghề Thủ Sỹ, bạn sẽ bắt gặp hình ảnh làng quê Bắc Bộ “thu nhỏ” với những ngôi nhà mái ngói đỏ, nếp nhà ba gian cổ kính, những lũy tre làng xanh mướt. Đặc biệt hơn, cuộc sống bình yên và giản dị tại đây sẽ giúp bạn xoa dịu tâm hồn, tạm quên đi những mệt mỏi, bộn bề nơi thành phố đông đúc. Đến làng Thủ Sỹ khi đi du lịch Hưng Yên sẽ giúp bạn được thư giãn hoàn toàn, những mệt mỏi và căng thẳng gần như biến mất.
\end{itemize}

\begin{itemize}
    \item{\textbf{Các thông tin khác - Cách di chuyển}} 
\end{itemize}

\section{Làng nghề tương bần Yên Nhân}
\begin{itemize}
    \item{\textbf{Địa chỉ}} Thị xã Mỹ Hào, tỉnh Hưng Yên

    \item{\textbf{Giờ mở cửa:}} Cả ngày

    \item{\textbf{Giá vé/ chi phí:}} Miễn phí

    \item{\textbf{Phân loại điểm du lịch}} 

    \item{\textbf{Nguồn thông tin}} 

    \item{\textbf{Thông tin về địa điểm:}} Làng nghề tương bần Yên Nhân nổi tiếng với nghề làm tương tồn tại hàng trăm năm, được truyền từ đời này sang đời khác. Khi vừa đặt chân đến ngôi làng này, bạn sẽ bắt gặp mùi thơm đặc trưng của tương. Mỗi chai tương ở đây đều được tự chế biến bởi các gia đình trong làng, đảm bảo yếu tố chất lượng, vệ sinh và an toàn tuyệt đối. Tương được nấu từ những nguyên liệu có nguồn gốc tự nhiên dễ kiếm nhưng để tương ngon, người dân địa phương đã chế biến cực kỳ khéo léo và tỉ mỉ, tạo nên mùi vị thơm ngon khác biệt so với các vùng khác. Khách du lịch đến đây thường mua tương về làm quà hoặc “dự trữ” để sử dụng cho những bữa cơm hàng ngày.
\end{itemize}

\begin{itemize}
    \item{\textbf{Các thông tin khác - Cách di chuyển}} 
\end{itemize}

\section{Grand World}
\begin{itemize}
    \item{\textbf{Địa chỉ}} Xã Nghĩa Trụ, huyện Văn Giang, tỉnh Hưng Yên

    \item{\textbf{Giờ mở cửa:}} Cả ngày

    \item{\textbf{Giá vé/ chi phí:}} Miễn phí

    \item{\textbf{Phân loại điểm du lịch}} 

    \item{\textbf{Nguồn thông tin}} 

    \item{\textbf{Thông tin về địa điểm:}} Đi du lịch Hưng Yên, một địa điểm mà bạn không nên bỏ lỡ đó là Grand World. Đến đây, du khách sẽ được đắm chìm trong khung cảnh châu Âu cổ điển với những ngôi nhà rực rỡ sắc màu, dòng kênh đào êm dịu mang đến bầu không khí lãng mạn. Hay được ghé thăm tháp đồng hồ cổ kính, đài phun nước Amore, quảng trường Grande…. Ở Grand World, mọi góc đều sở hữu thiết kế đẹp để bạn tha hồ chụp ảnh “sống ảo”. Không chỉ dừng lại ở đó, Grand World còn được mệnh danh là thiên đường ẩm thực và mua sắm với sự góp mặt của nhiều thương hiệu khác nhau, mang đến những trải nghiệm tuyệt vời cho du khách. Grand World là địa điểm du lịch Hưng Yên nổi bật nhất hiện nay.
\end{itemize}

\begin{itemize}
    \item{\textbf{Các thông tin khác - Cách di chuyển}} 
\end{itemize}

\section{Cánh đồng hoa Cúc Chi}
\begin{itemize}
    \item{\textbf{Địa chỉ}} Thôn Nghĩa Trai, xã Tân Quang, huyện Văn Lâm, tỉnh Hưng Yên

    \item{\textbf{Giờ mở cửa:}} Cả ngày

    \item{\textbf{Giá vé/ chi phí:}} Miễn phí

    \item{\textbf{Phân loại điểm du lịch}} 

    \item{\textbf{Nguồn thông tin}} 

    \item{\textbf{Thông tin về địa điểm:}} Đi du lịch Hưng Yên vào dịp cuối năm, bạn hãy ghé đến cánh đồng hoa cúc chi để chiêm ngưỡng bức tranh thiên nhiên tuyệt đẹp. Hoa cúc chi, hay còn được biết đến với tên gọi khác là hoa kim cúc, hoa hoàng cúc, đây chính là biểu tượng của Phố Hiến khi mùa đông về. Chỉ mất khoảng một giờ di chuyển bằng xe máy từ trung tâm Hà Nội, bạn sẽ được đến với cánh đồng hoa cúc chi màu vàng rực rỡ, sáng bừng cả một vùng quê thơ mộng và yên bình. Địa điểm này thực sự rất phù hợp với những du khách đam mê chụp hình check in “sống ảo”. Cánh đồng hoa cúc chi là địa điểm “sống ảo” tuyệt đẹp dành cho khách du lịch Hưng Yên.
\end{itemize}

\begin{itemize}
    \item{\textbf{Các thông tin khác - Cách di chuyển}} 
\end{itemize}

\section{Mega Grand World}
\begin{itemize}
    \item{\textbf{Địa chỉ}} Xã Nghĩa Trụ, H. Văn Giang, Hưng Yên.

    \item{\textbf{Giờ mở cửa:}} Cả ngày.

    \item{\textbf{Giá vé/ chi phí:}} Miễn phí.

    \item{\textbf{Phân loại điểm du lịch}} 

    \item{\textbf{Nguồn thông tin}} 

    \item{\textbf{Thông tin về địa điểm:}} Mega Grand World Hà Nội tái hiện khung cảnh châu Âu cổ kính, đưa du khách lạc vào một Venice đầy thơ mộng với những ngôi nhà rực rỡ sắc màu và dòng kênh đào yên ả. Nổi bật ở ngay đầu là cây cầu đi bộ Đông Tây nối liền 2 khu vực Venice và phố K-Town, nơi mọi người được trải nghiệm văn hóa Hàn Quốc đầy sôi động cũng như mua sắm thả ga. Grand World là thiên đường sống ảo với những góc phố, cung đường đẹp như tranh vẽ, từ tháp đồng hồ cổ kính, quảng trường Grande đến đài phun nước Amore. Bên cạnh đó, nơi đây còn là thiên đường ẩm thực và mua sắm với các thương hiệu nổi tiếng như: Adidas, BOO, F1 Enter, Áo dài Sài Gòn Hà Anh. Grand World là thiên đường ẩm thực với nhiều lựa chọn hấp dẫn, từ món ăn đường phố, truyền thống đến các nhà hàng như Nét Huế, TexGrill, Bò Tơ Quán Mộc. Không thể thiếu những thương hiệu cà phê nổi tiếng như: Trung Nguyên, Katinat, Highlands, mang đến trải nghiệm trọn vẹn cho du khách. Grand World về đêm còn mang đến vẻ đẹp xa hoa, tráng lệ với vở diễn thực cảnh mang tên “The Grand Voyage”. Sử dụng công nghệ mapping 3D tiên tiến, show diễn tái hiện một hành trình giao thương đầy ấn tượng, bắt đầu từ châu Âu, đi qua con đường tơ lụa huyền thoại và khám phá vùng đất phía biển Đông, với điểm dừng chân là thương cảng Hội An – Việt Nam trong truyền thuyết.
\end{itemize}

\begin{itemize}
    \item{\textbf{Các thông tin khác - Cách di chuyển}} 
\end{itemize}

\section{VinWonders Wave Park \& Water Park}
\begin{itemize}
    \item{\textbf{Địa chỉ}} Đường San Hô, Vinhomes Ocean Park 2, X. Nghĩa Trụ, H. Văn Giang, Hưng Yên.

    \item{\textbf{Giờ mở cửa:}} 09h00 – 12h00 và 14h00 – 18h00.

    \item{\textbf{Giá vé/ chi phí:}} Miễn phí.

    \item{\textbf{Phân loại điểm du lịch}} 

    \item{\textbf{Nguồn thông tin}} 

    \item{\textbf{Thông tin về địa điểm:}} VinWonders Wave Park \& Water Park là một điểm đến giải trí hấp dẫn trong hành trình du lịch Hưng Yên. Nơi đây được mệnh danh là thiên đường vui chơi với vô số hoạt động thú vị, cùng cảnh quan tuyệt đẹp của biển nhân tạo, bờ cát trắng trải dài và hồ nước mặn rộng lớn. Điều đặc biệt, địa điểm du lịch Văn Giang Hưng Yên này chỉ cách các quận nội đô Hà Nội khoảng 30 phút đi xe, dễ dàng đến thông qua mạng lưới xe buýt điện Vinbus dày đặc.
    Nằm trong lòng khu đô thị Vinhomes Ocean Park 2, VinWonders Wave Park hứa hẹn mang đến cho bạn một kỳ nghỉ biển trong mơ. Hãy đến vui chơi thả ga trong làn nước mát, ngồi thư giãn trên bãi cát nhân tạo trắng muốt, ở đây có tận 6 bể tạo sóng để khám phá, thử sức cùng 10 trò chơi nước hấp dẫn và trải nghiệm các hoạt động thể thao biển đầy thú vị như: chèo kayak, mô tô nước, lướt sóng,...
    VinWonders Water Park là một công viên nước quy mô lớn, mang đến hàng loạt trải nghiệm giải trí hấp dẫn. Từ hệ thống đường trượt hiện đại, các sân chơi trẻ em sinh động, bể bơi lớn ngoài trời cho đến bể bơi bốn mùa trong nhà. Khu vui chơi này còn cung cấp đầy đủ những tiện ích cũng như dịch vụ phục vụ cho vui chơi và nghỉ dưỡng, để đáp ứng nhu cầu đa dạng của khách hàng.
\end{itemize}

\begin{itemize}
    \item{\textbf{Các thông tin khác - Cách di chuyển}} 
\end{itemize}

\section{Làng hoa Xuân Quan}
\begin{itemize}
    \item{\textbf{Địa chỉ}} Xuân Quan, H. Văn Giang, Hưng Yên.

    \item{\textbf{Giờ mở cửa:}} Cả ngày.

    \item{\textbf{Giá vé/ chi phí:}} Miễn phí.

    \item{\textbf{Phân loại điểm du lịch}} 

    \item{\textbf{Nguồn thông tin}} 

    \item{\textbf{Thông tin về địa điểm:}} Xã Xuân Quan sử dụng đến hơn 60\% diện tích để phục vụ cho trồng hoa và cây cảnh, tổng cộng 900 hộ dân làm nghề chuyên nghiệp. Nhờ ứng dụng công nghệ cao, nhiều hộ nông dân đã đạt thu nhập cao từ nghề trồng hoa, góp phần mang lại mùa xuân thịnh vượng cho cộng đồng. Nơi đây nổi tiếng với đa dạng các loại hoa, từ hồng ngoại, pansy, thược dược, ly, đào, đỗ quyên, trà cổ, đồng tiền, cẩm chướng đến hải đường, trạng nguyên,... tạo nên vẻ độc đáo cho làng hoa. Địa điểm du lịch Văn Giang Hưng Yên này có nhiều khác biệt với một số làng hoa tại miền Bắc ở phong cách trồng hoa độc đáo. Thay vì chọn trồng hoa trên đất như truyền thống, người dân ở Xuân Quan lại chọn phương pháp trồng hoa bằng chậu treo. Cách làm này không chỉ mang lại hiệu quả kinh tế cao mà còn tạo nên vẻ đẹp lôi cuốn, thu hút du khách. Đến làng Xuân Quan, khung cảnh những cây cảnh, giỏ hoa rực rỡ sắc màu được treo lủng lẳng sẽ khiến bạn không khỏi ngỡ ngàng, không chỉ vào dịp Tết Nguyên đán mà ở mọi thời điểm trong năm. Đặc biệt là dịp gần Tết, khách du lịch đến đây để tận hưởng không khí rộn ràng tràn ngập sắc màu tươi tắn. Nơi đây gần làng gốm Bát Tràng, thuận tiện cho du khách mua sắm và tạo không khí Tết trọn vẹn. Lễ hội làng hoa diễn ra 3 năm một lần, mang đến cơ hội tham quan triển lãm hoa, cây cảnh, chiêm ngưỡng tiểu cảnh độc đáo và học hỏi về nông sản địa phương.
\end{itemize}

\begin{itemize}
    \item{\textbf{Các thông tin khác - Cách di chuyển}} 
\end{itemize}

\section{Ecopark}
\begin{itemize}
    \item{\textbf{Địa chỉ}} 3 xã Xuân Quan – Phụng Công – Cửu Cao, huyện Văn Giang, Hưng Yên.

    \item{\textbf{Giờ mở cửa:}} Cả ngày.

    \item{\textbf{Giá vé/ chi phí:}} Miễn phí.

    \item{\textbf{Phân loại điểm du lịch}} 

    \item{\textbf{Nguồn thông tin}} 

    \item{\textbf{Thông tin về địa điểm:}} Ecopark với diện tích hơn 100 ha cây xanh và hồ nước, là "lá phổi xanh" của ngoại thành Hà Nội, mang đến không gian sống trong lành, gần gũi thiên nhiên. Ecopark thuộc top những khu đô thị xanh đẹp cũng như đáng sống hàng đầu tại nước ta, không những thế còn sở hữu vị trí đắc địa, xung quanh là sông Bắc Hưng Hải, hội tụ đầy đủ “vượng lộc, vượng khí và vượng thủy”.
    Các công trình nổi bật trong khu đô thị Ecopark gồm:
    \begin{itemize}
        \item{Công viên Hồ Thiên Nga rộng 100ha trong khu đô thị Ecopark là điểm đến lý tưởng cho các hoạt động dạo chơi, thư giãn cùng gia đình. Nơi đây có hàng nghìn cây xanh, tiếng chim hót líu lo và hồ Thiên Nga với 20 chú thiên nga xinh đẹp. Du khách có thể mang theo bánh mì để cho thiên nga ăn.}
        \item{Công viên Mùa Hạ rộng 2,5ha, là địa điểm lý tưởng cho các hoạt động dã ngoại, ngoại khóa hay team building.}
        \item{Công viên Mùa Thu rộng 3,5 ha, tràn ngập sắc xanh của cây cối.}
        \item{Công viên Mùa Xuân với diện tích 3ha nổi bật bởi những khóm hoa đua nhau khoe sắc, đồng thời cũng là nơi tổ chức nhiều sự kiện ngoài trời.}
        \item{Khu vui chơi ngoài trời bằng gỗ này, do tổ chức giáo dục Nhật Bản thiết kế, là nơi trẻ em được tự do khám phá, vận động và rèn luyện sức khỏe.}
        \item{Con đường tràn ngập hoa vàng cỏ xanh xung quanh khu biệt thự đảo Ecopark Grand.}
        \item{Thiên đường ẩm thực – mua sắm: địa điểm du lịch Văn Giang Hưng Yên này là thiên đường ẩm thực và mua sắm, thu hút cư dân và du khách với đa dạng món ăn vùng miền và các nhà hàng, quán cafe đẹp mắt, gần gũi thiên nhiên.}
    \end{itemize}
\end{itemize}

\begin{itemize}
    \item{\textbf{Các thông tin khác - Cách di chuyển}} 
\end{itemize}

\section{Chùa Mễ Sở}
\begin{itemize}
    \item{\textbf{Địa chỉ}} thôn Mễ Sở, Mễ Sở, Văn Giang, Hưng Yên.

    \item{\textbf{Giờ mở cửa:}} Cả ngày.

    \item{\textbf{Giá vé/ chi phí:}} Miễn phí.

    \item{\textbf{Phân loại điểm du lịch}} Du lịch tâm linh.

    \item{\textbf{Nguồn thông tin}} 

    \item{\textbf{Thông tin về địa điểm:}} Chùa Mễ Sở thuộc hệ phái Bắc Tông, chùa được xây dựng vào thời Hậu Lê và đã trải qua nhiều lần trùng tu. Hiện nay, trong chùa đang gìn giữ tượng Bồ Tát Quan Âm Thiên Thủ Thiên Nhãn được làm bằng gỗ. Tượng Quan Âm cao 2,80m và bệ tượng cao 0,53m. Điểm du lịch tâm linh ở Hưng Yên này là nơi lưu giữ bức tượng Quan Âm Thiên Thủ Thiên Nhãn độc bản, được tạo tác tinh xảo bằng kỹ thuật điêu khắc thủ công truyền thống. Pho tượng cổ này là một trong ba kiệt tác đỉnh cao của nghệ thuật tạo tác tượng Quan Âm Thiên Thủ Thiên Nhãn cổ ở Việt Nam, đại diện của nền mỹ thuật vào khoảng đầu thế kỷ 19. Hằng năm, chùa Mễ Sở thu hút đông đảo du khách thập phương gần xa về chiêm bái và lễ Phật. Nếu có dịp về với Văn Giang để khám phá, bạn đừng bỏ qua cơ hội ghé thăm chùa Mễ Sở để được tận mắt chiêm ngưỡng Bảo vật Quốc gia hiếm có lại mang nhiều ý nghĩa tâm linh và giá trị văn hóa đặc sắc của vùng đất giàu truyền thống văn hiến này.
\end{itemize}

\begin{itemize}
    \item{\textbf{Các thông tin khác - Cách di chuyển}} 
\end{itemize}

\section{Chùa Phú Thị}
\begin{itemize}
    \item{\textbf{Địa chỉ}} thôn Phú Thị, Văn Giang, Hưng Yên.

    \item{\textbf{Giờ mở cửa:}} Cả ngày.

    \item{\textbf{Giá vé/ chi phí:}} Miễn phí.

    \item{\textbf{Phân loại điểm du lịch}} Du lịch tâm linh.

    \item{\textbf{Nguồn thông tin}} 

    \item{\textbf{Thông tin về địa điểm:}} Chùa Phú Thị, còn gọi là chùa Hưng Phúc, tọa lạc tại thôn Phú Thị, được xây dựng vào thời Hậu Lê và được trùng tu nhiều lần, gần đây nhất là năm 1991. Kiến trúc của địa điểm du lịch Văn Giang Hưng Yên này theo kiểu chữ Đinh, cửa hướng Tây Nam, với bệ kèo gỗ hình càng cua và trần gỗ cuốn vòm tạo nên không gian trầm tĩnh, uy nghi cho chốn linh thiêng. Tiền đường 5 gian cao ráo và thoáng mát, với kiến trúc kiểu chồng diêm, có bốn hàng cột lim bằng nhau. Trên xà ngang, cửa võng chạm nổi "lưỡng long chầu nguyệt" sơn son thếp vàng tôn lên vẻ uy nghi. Điện Phật được bài trí trang nghiêm, chùa còn lưu giữ nhiều pho tượng cổ bằng gỗ, đất nung phủ sơn. Bốn pho tượng "ông Thiện, ông Ác, Thần Sấm, Thần Sét" được đặt giáp tường, tăng thêm vẻ uy linh cho tiền đường. Nối tiếp tiền đường là 4 gian hậu cung. Chùa lưu giữ nhiều di tích và pho tượng cổ quý giá. Nhân dân trong vùng còn dành riêng một nơi tôn nghiêm để thờ tiến sĩ Chu Mạnh Trinh, người có công trùng tu và xây dựng lại ngôi chùa này. Năm 1984, chùa được Bộ Văn hóa - Thông tin công nhận là Di tích Lịch sử - Văn hóa Quốc gia.
\end{itemize}

\begin{itemize}
    \item{\textbf{Các thông tin khác - Cách di chuyển}} 
\end{itemize}

\section{Đình Đa Ngưu - Đình trăm cột}
\begin{itemize}
    \item{\textbf{Địa chỉ}} X. Tân Tiến, H. Văn Giang, Hưng Yên.

    \item{\textbf{Giờ mở cửa:}} Cả ngày.

    \item{\textbf{Giá vé/ chi phí:}} Miễn phí.

    \item{\textbf{Phân loại điểm du lịch}} Du lịch tâm linh.

    \item{\textbf{Nguồn thông tin}} 

    \item{\textbf{Thông tin về địa điểm:}} Đình Đa Ngưu hay còn gọi là "đình trăm cột", nằm giữa làng Đa Ngưu là một ngôi đình cổ giữ được khá nguyên vẹn kiến trúc và di vật quý giá. Theo truyền thuyết, làng Đa Ngưu có địa thế đẹp, thoáng mát, được xem là "Hình nhân quái bảng", nên người dân đã xây dựng ngôi đình này tại đây. Địa điểm du lịch Văn Giang Hưng Yên với tuổi đời gần 7 thế kỷ, mang phong cách kiến trúc thời Lý - Trần, được thiết kế theo hình chữ "Sĩ" gồm hai tòa tiền tế và hậu cung, tượng trưng cho "học trò và người trí thức". Ngôi đình còn được gọi là "đình trăm cột" bởi có 100 cây cột gỗ, được kết nối bằng kỹ thuật đố mộng tinh xảo, không dùng đinh. Ban đầu các thợ đều lắc đầu nhưng về sau, có một thợ đến nhận làm nhưng chỉ xin dựng 100 cột, 1 cây còn lại thì chẻ ra làm tông đục. Đình thờ Đức Thánh Chử Đồng Tử, một trong "Tứ bất tử" của tín ngưỡng Việt Nam, cùng hai vị phu nhân Tiên Dung và Tây Sa công chúa. Đình được xây dựng vào năm 1706, trùng tu năm 1910 theo kiến trúc tiền Nhất hậu Đinh, năm 1910 thời Nguyễn trùng tu lại gồm 3 tòa với các chi tiết chạm trổ tinh vi. Đình Đa Ngưu sở hữu kiến trúc đồ sộ và được điêu khắc hết sức tinh xảo. Đình còn nổi tiếng khi sở hữu đến 100 cây cột bằng gỗ lim hiếm có, được xếp hạng là Di tích Quốc gia năm 1995. Đình Đa Ngưu là nơi nét tài hoa của những người thợ Việt xưa được lưu giữ, hiếm ngôi đình nào được dựng 100 cột từ 101 cây gỗ như nơi này. Hàng năm, hội làng diễn ra vào ngày 12-15/2 âm lịch, với nghi thức tắm rửa các ngai thờ. Du khách đến thăm đình vào mùa hè sẽ được thư giãn trong không gian tĩnh lặng và hương sen thơm ngát.
\end{itemize}

\begin{itemize}
    \item{\textbf{Các thông tin khác - Cách di chuyển}} 
\end{itemize}


\newpage
\section*{{Ẩm thực}}
\setcounter{section}{0}



\end{document}