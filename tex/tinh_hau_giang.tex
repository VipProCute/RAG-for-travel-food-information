\documentclass{article}
\usepackage[vietnamese]{babel}
\usepackage[letterpaper,top=1cm,bottom=1cm,left=1.5cm,right=1.5cm,marginparwidth=1.75cm]{geometry}
\usepackage{amsmath}
\usepackage{graphicx}
\usepackage[colorlinks=true, allcolors=blue]{hyperref}
\title{Cẩm nang du lịch Tỉnh Hậu Giang}

\begin{document}
\begin{center}
    \fontsize{18}{20}\textbf{Cẩm nang du lịch Tỉnh Hậu Giang}
\end{center}
\begin{abstract}
    (Giới thiệu)
\end{abstract}
\section*{Điểm du lịch}

\section{Chợ nổi Ngã Bảy – Phụng Hiệp}
\begin{itemize}
    \item{\textbf{Địa chỉ}} RRHM+8PR, Đại Thành, Ngã Bảy
    \item{\textbf{Giờ mở cửa:}} 03:00 - 09:00
    \item{\textbf{Giá vé/ chi phí:}} Miễn phí
    \item{\textbf{Phân loại điểm du lịch} }
    \item{\textbf{Nguồn thông tin}}
    \item{\textbf{Thông tin về địa điểm:}}
\end{itemize}
Chợ nổi Ngã Bảy - Phụng Hiệp không chỉ là một trung tâm buôn bán sầm uất mà còn là bức tranh sống động về văn hóa đặc trưng của vùng đồng bằng sông Cửu Long. Nằm ở điểm giao nhau của bảy con sông quanh co, chợ nổi này mang đến một không gian mua sắm độc đáo trên mặt nước, nơi mà mỗi tiếng cười đều phản ánh nét đặc trưng của cuộc sống miền sông nước. Từ rất sớm, khoảng 2-3 giờ sáng mỗi ngày, hàng loạt thuyền từ khắp nơi sẽ tụ họp về đây để khởi đầu cho một ngày buôn bán nhộn nhịp.
Chợ nổi Ngã Bảy - Phụng Hiệp không chỉ bày bán nông sản và thủy hải sản tươi ngon, mà còn có cả các quầy hàng ẩm thực đặc trưng của vùng đồng bằng sông Cửu Long như bún, hủ tiếu, bánh canh... Đặc biệt, không thể không kể đến những quầy hàng cà phê nổi trên sông, nơi mà du khách có thể thưởng thức ly cà phê đặc biệt trong không gian yên bình của bình minh sông nước. Chợ nổi cách trung tâm Cần Thơ khoảng 30km, trở thành địa điểm du lịch Hậu Giang lý tưởng cho những ai muốn khám phá văn hóa và phong cách sống độc đáo của người dân miền Tây.

\begin{itemize}
    \item{\textbf{Các thông tin khác - Cách di chuyển}}
\end{itemize}

\section{Khu du lịch sinh thái Tây Đô}
\begin{itemize}
    \item{\textbf{Địa chỉ}} Ấp Tân Long, Xã Tân Bình, Huyện Phụng Hiệp, Bình Thành
    \item{\textbf{Giờ mở cửa:}} 05:45–22:00
    \item{\textbf{Giá vé/ chi phí:}} Miễn phí
    \item{\textbf{Phân loại điểm du lịch} }
    \item{\textbf{Nguồn thông tin}}
    \item{\textbf{Thông tin về địa điểm:}}
\end{itemize}
Khu du lịch sinh thái Tây Đô cũng là một điểm đến thú vị mà bạn không muốn bỏ lỡ nếu du lịch Hậu Giang. Không gian nơi đây khá là xanh mát, yên bình với cảnh sắc thiên nhiên tươi mới.
Bước chân vào khu du lịch Tây Đô, bạn sẽ được đắm chìm trong không gian xanh của hàng loạt cây cỏ, hoa lá cùng tiếng ca líu lo của các loài chim. Tất cả những điểm đặc sắc trên đã tạo nên một bức tranh thiên nhiên sống động, đầy màu sắc cho nơi đây. Ngoài ra, khu du lịch Hậu Giang này cũng cung cấp nhiều hoạt động và giải trí đặc sắc cho bạn tha hồ trải nghiệm, từ các khu vui chơi trẻ em, người lớn đến các hoạt động mang tính gắn kết gia đình. Những nhà hàng và quán ăn tại khu du lịch Tây Đô không chỉ phục vụ các món ăn truyền thống đặc sắc của vùng đất này mà còn mang đến cho bạn cơ hội thưởng thức những món ăn hiện đại với đa dạng các loại nguyên liệu cùng hình thức chế biến.

\begin{itemize}
    \item{\textbf{Các thông tin khác - Cách di chuyển}}
\end{itemize}

\section{Khu bảo tồn thiên nhiên Lung Ngọc Hoàng}
\begin{itemize}
    \item{\textbf{Địa chỉ}} Ấp Mùa Xuân, xã Phương Bình, huyện Phụng Hiệp
    \item{\textbf{Giờ mở cửa:}} 24/24
    \item{\textbf{Giá vé/ chi phí:}} 60.000 VNĐ/ Vé
    \item{\textbf{Phân loại điểm du lịch} }
    \item{\textbf{Nguồn thông tin}}
    \item{\textbf{Thông tin về địa điểm:}}
\end{itemize}
Khu bảo tồn thiên nhiên Lung Ngọc Hoàng nằm ẩn mình giữa lòng tỉnh Hậu Giang và cũng là điểm đến lý tưởng cho những ai yêu mến và muốn khám phá vẻ đẹp hoang sơ của thiên nhiên. Với tên gọi mang ý nghĩa “vùng rừng lầy hoang dã của Ông Trời”, khu bảo tồn này thực sự là một bức tranh thiên nhiên sống động, đầy màu sắc bậc nhất tỉnh Hậu Giang.
Bước chân vào khu bảo tồn, bạn sẽ được chiêm ngưỡng những con kênh dài xanh mướt, nơi mà bản hòa nhạc của thiên nhiên bắt đầu qua tiếng chảy rì rào của nước, tiếng xào xạc của lá cây và tiếng ca ríu rít của các loài chim. Lung Ngọc Hoàng cũng là nơi trú ngụ của nhiều loài động - thực vật quý hiếm, điều này giúp khu bảo tồn trở thành một trong những điểm đến quan trọng trên bản đồ du lịch sinh thái của Việt Nam.

\begin{itemize}
    \item{\textbf{Các thông tin khác - Cách di chuyển}}
\end{itemize}

\section{Rừng tràm chim Vị Thủy}
\begin{itemize}
    \item{\textbf{Địa chỉ}} xã Vĩnh Tường, huyện Vị Thủy
    \item{\textbf{Giờ mở cửa:}} 6:00 - 16:00
    \item{\textbf{Giá vé/ chi phí:}} 120.000 VNĐ/ vé
    \item{\textbf{Phân loại điểm du lịch} }
    \item{\textbf{Nguồn thông tin}}
    \item{\textbf{Thông tin về địa điểm:}}
\end{itemize}
Khu du lịch Sinh thái rừng tràm chim Vị Thủy với tổng diện tích lên đến 145ha là một trong những khu du lịch sinh thái lớn và độc đáo nhất tại Hậu Giang. Không chỉ là một địa điểm du lịch, rừng tràm chim Vị Thủy còn thể hiện nét đẹp văn hóa đặc trưng của con người nơi đây.
Khi đặt chân đến khu du lịch sinh thái này, bạn sẽ được trải nghiệm những chuyến xuôi thuyền trên những dòng kênh mướt mát dưới những tán cây xanh. Nhờ vậy bạn sẽ cảm nhận được vẻ đẹp yên bình, thơ mộng của thiên nhiên nơi đây. Phải nói rằng cảnh vật xung quanh rừng tràm chim Vị Thủy đẹp đến nao lòng và bạn nhất định phải đến đây dù chỉ một lần.

\begin{itemize}
    \item{\textbf{Các thông tin khác - Cách di chuyển}}
\end{itemize}

\section{Du lịch Vị Thanh}
\begin{itemize}
    \item{\textbf{Địa chỉ}} huyện Vị Thanh, Hậu Giang
    \item{\textbf{Giờ mở cửa:}}
    \item{\textbf{Giá vé/ chi phí:}}
    \item{\textbf{Phân loại điểm du lịch} }
    \item{\textbf{Nguồn thông tin}}
    \item{\textbf{Thông tin về địa điểm:}}
\end{itemize}
Vị Thanh - một huyện của tỉnh Hậu Giang và cũng là nơi hội tụ bản sắc văn hóa độc đáo và nhiều địa điểm tham quan hấp dẫn. Vị Thanh không chỉ là nơi gắn liền với lịch sử hào hùng của đất nước mà còn là điểm đến lý tưởng cho những ai yêu thích khám phá và trải nghiệm.
Một trong những điểm đến nổi bật nhất của Vị Thanh chính là Công viên ánh sáng kỳ quan cổ đại The Miracle. Với diện tích lên đến 7.180 m2, công viên tái hiện 7 kỳ quan nổi tiếng thế giới chắc chắn sẽ mang đến cho bạn cơ hội ngắm nhìn và tìm hiểu về những kiệt tác kiến trúc lừng lẫy từ Babylon đến Alexandria ngay tại lòng Hậu Giang.
Ngoài ra, khi đến du lịch Vị Thanh, bạn cũng sẽ không thể bỏ qua khu du lịch Khóm Cầu Đúc, nơi mang đến trải nghiệm thú vị khi ngồi trên chiếc xuồng nhỏ len lỏi vào từng thửa ruộng khóm vàng ươm, xanh mát. Đặc biệt, vào mùa thu hoạch, cảnh buôn bán khóm tấp nập sẽ mang đến cho bạn cái nhìn sâu sắc về đời sống người dân nơi đây.
Công viên Chiến Thắng và Công viên Xà No cũng là những điểm tham quan thú vị trong chuyến hành trình du lịch Hậu Giang của bạn. Nơi đây tạo cho bạn cơ hội thả mình dưới bóng mát cây xanh, hóng gió bên bờ kè và ngắm nhìn hoàng hôn dần buông xuống trên sông, tạo nên bức tranh thiên nhiên thơ mộng. Khi màn đêm buông xuống, Chợ đêm Vị Thanh sẽ mang đến không khí nhộn nhịp với 45 gian hàng bán đồ ăn, quà lưu niệm, đặc sản địa phương và quần áo cực kỳ sinh động. Có thể nói, với vẻ đẹp yên bình, bản sắc văn hóa phong phú và nhiều địa điểm tham quan độc đáo, Vị Thanh chắc chắn sẽ mang đến cho bạn những trải nghiệm du lịch khó quên.

\begin{itemize}
    \item{\textbf{Các thông tin khác - Cách di chuyển}}
\end{itemize}

\section{Khu du lịch Sinh thái Phú Hữu}
\begin{itemize}
    \item{\textbf{Địa chỉ}} 925, Phú Hữu, Châu Thành
    \item{\textbf{Giờ mở cửa:}} 24/24
    \item{\textbf{Giá vé/ chi phí:}} 100.000 – 350.000 VNĐ/ vé
    \item{\textbf{Phân loại điểm du lịch} }
    \item{\textbf{Nguồn thông tin}}
    \item{\textbf{Thông tin về địa điểm:}}
\end{itemize}
Khu du lịch Sinh thái Phú Hữu mang đến cho du khách cơ hội khám phá không gian trong lành của miền quê sông nước cũng như tham gia vào nhiều hoạt động vui chơi giải trí từ dân gian đến hiện đại.
Một trong những điểm nổi bật của khu du lịch này là hệ thống chuỗi khách sạn tiện nghi và hệ thống căn hộ Homestay miền quê. Những căn hộ Homestay được xây dựng công phu bằng tre lá, mang đến cho du khách cảm giác ấm cúng, gần gũi với thiên nhiên.
Đến đây, du khách sẽ được tham gia vào nhiều hoạt động dân dã như trồng rau, câu cá, chăm sóc vườn hoa, mang đến trải nghiệm thú vị về lối sống thôn quê của người dân Nam Bộ.

\begin{itemize}
    \item{\textbf{Các thông tin khác - Cách di chuyển}}
\end{itemize}

\section{Thiền viện Trúc Lâm Hậu Giang}
\begin{itemize}
    \item{\textbf{Địa chỉ}} Vĩnh Tường, Long Mỹ
    \item{\textbf{Giờ mở cửa:}} 7:00 – 21:00
    \item{\textbf{Giá vé/ chi phí:}} Miễn phí
    \item{\textbf{Phân loại điểm du lịch} }
    \item{\textbf{Nguồn thông tin}}
    \item{\textbf{Thông tin về địa điểm:}}
\end{itemize}

Khánh thành vào năm 2018, Thiền viện Trúc Lâm Hậu Giang nhanh chóng trở thành điểm thu hút du khách tham quan và chiêm bái. Thiền viện này cũng tuân theo Thiền phái Trúc Lâm Yên Tử. Nơi này không chỉ là nơi sinh hoạt tôn giáo của các tăng ni, phật tử mà còn là công trình đồ sộ với kiến trúc độc đáo, mang đậm phong cách kiến trúc Phật giáo truyền thống.

\begin{itemize}
    \item{\textbf{Các thông tin khác - Cách di chuyển}}
\end{itemize}

\section{Chùa Aranhứt}
\begin{itemize}
    \item{\textbf{Địa chỉ}} xã Đông Thạnh, huyện Châu Thành A
    \item{\textbf{Giờ mở cửa:}}
    \item{\textbf{Giá vé/ chi phí:}} Miễn phí
    \item{\textbf{Phân loại điểm du lịch} }
    \item{\textbf{Nguồn thông tin}}
    \item{\textbf{Thông tin về địa điểm:}}
\end{itemize}

Chùa Aranhứt tọa lạc tại huyện Châu Thành A và được biết đến là ngôi chùa Khmer cổ nhất tại Hậu Giang lẫn ở Đồng bằng sông Cửu Long. Theo lịch sử, vào khoảng năm 1632, người Khmer ở địa phương đã quyết định xây dựng ngôi chùa này như một nơi tín ngưỡng tôn giáo và sinh hoạt tinh thần vào các dịp lễ hội. Kiến trúc của chùa Aranhứt mang đậm nét đặc trưng của kiến trúc Khmer với những nét chạm trổ và điêu khắc độc đáo. Với bề dày lịch sử, chùa Aranhứt không chỉ là điểm nhấn văn hóa tín ngưỡng mà còn có tiềm năng phát triển thành điểm du lịch Hậu Giang, thu hút khách du lịch đến tham quan và tìm hiểu về văn hóa tâm linh của khu vực này.

\begin{itemize}
    \item{\textbf{Các thông tin khác - Cách di chuyển}}
\end{itemize}

\section{Căn cứ Bà Bái}
\begin{itemize}
    \item{\textbf{Địa chỉ}} Phương Bình, Phụng Hiệp, Hậu Giang
    \item{\textbf{Giờ mở cửa:}} 7:00 - 20:00
    \item{\textbf{Giá vé/ chi phí:}} Miễn phí
    \item{\textbf{Phân loại điểm du lịch} }
    \item{\textbf{Nguồn thông tin}}
    \item{\textbf{Thông tin về địa điểm:}}
\end{itemize}

Căn cứ Bà Bái hay còn gọi Căn cứ tỉnh ủy Cần Thơ là một trong những điểm du lịch Hậu Giang mang tính lịch sử mà du khách không nên bỏ qua khi đến vùng đất Tây Đô. Đây là nơi ghi dấu những diễn biến lịch sử quan trọng của nhân dân Cần Thơ trong giai đoạn 1972 – 1975, qua đó giúp du khách hiểu rõ hơn về quá trình chiến đấu anh dũng của nhân dân Việt Nam trong thời kỳ kháng chiến chống Mỹ.

Tại đây, du khách sẽ được chiêm ngưỡng sa bàn căn cứ tỉnh ủy Cần Thơ, cùng với nhiều hiện vật và hình ảnh về chiến tranh, tái hiện lại không khí và hoàn cảnh của những năm tháng khó khăn và đầy thách thức của cha ông ta.

\begin{itemize}
    \item{\textbf{Các thông tin khác - Cách di chuyển}}
\end{itemize}

\section{Công viên Giải trí Kittyd \& Minnied}
\begin{itemize}
    \item{\textbf{Địa chỉ}} QL1A, Tân Phú Thạnh, Châu Thành A
    \item{\textbf{Giờ mở cửa:}} 8:00 – 18:00
    \item{\textbf{Giá vé/ chi phí:}} 40.000 – 80.000 VNĐ/ vé vào cổng
    \item{\textbf{Phân loại điểm du lịch} } Công viên giải trí
    \item{\textbf{Nguồn thông tin}}
    \item{\textbf{Thông tin về địa điểm:}} Công viên giải trí Kittyd \& Minnied tọa lạc trong khuôn viên Trường Đại học Võ Trường Toản là điểm đến lý tưởng cho những ai yêu thích không gian giải trí hiện đại và sôi động. Với vị trí trọng điểm chỉ cách trung tâm TP Cần Thơ 7km, công viên này mang đến cho du khách một không gian giải trí lớn nhất và hiện đại bậc nhất khu vực đồng bằng sông Cửu Long. Trải rộng trên diện tích hơn 20 ha, Kittyd \& Minnied được thiết kế theo mô hình công viên giải trí hiện đại với nhiều tòa nhà, công trình mang phong cách kiến trúc châu u độc đáo. Du khách sẽ được thưởng ngoạn những kiến trúc hàng đầu thế giới được mô phỏng một cách tinh tế như tượng Nữ thần Tự do của Mỹ, tháp nghiêng Pisa của Italy, tháp Big Ben của Anh, Khải Hoàn Môn của Pháp,... Tại Kittyd \& Minnied, bạn sẽ được trải nghiệm không giới hạn số lần chơi tại tất cả các khu trò chơi của công viên, từ khu trò chơi cảm giác mạnh, khu trò chơi điện tử đến công viên nước hiện đại. Không chỉ có vậy, du khách còn được hòa mình vào không gian cổ tích với nhiều chương trình biểu diễn đặc sắc như diễu hành đường phố tái hiện lại các nhân vật cổ tích, vũ điệu samba sôi động hay thế giới cổ tích thần tiên tại Lâu đài Coeus với quần thể tượng thần Hy Lạp cổ đại. Công viên giải trí Kittyd \& Minnied không chỉ là nơi giải trí lý tưởng, mà còn là nơi du khách có thể tận hưởng không khí lãng mạn giữa lòng miền Tây sông nước, mang đến cho bạn những kỷ niệm khó quên trong chuyến du lịch Hậu Giang của mình.
\end{itemize}

\begin{itemize}
    \item{\textbf{Các thông tin khác - Cách di chuyển}}
\end{itemize}

\section{Khu du lịch sinh thái Tầm Vu}
\begin{itemize}
    \item{\textbf{Địa chỉ}} QL 61, Thạnh Hòa, Châu Thành A
    \item{\textbf{Giờ mở cửa:}} 6:00 – 22:00
    \item{\textbf{Giá vé/ chi phí:}} Miễn phí
    \item{\textbf{Phân loại điểm du lịch} } Khu du lịch sinh thái
    \item{\textbf{Nguồn thông tin}}
    \item{\textbf{Thông tin về địa điểm:}} Khu du lịch sinh thái Tầm Vu mang đến cho du khách một trải nghiệm du lịch độc đáo giữa lòng thiên nhiên xanh mát, kết hợp với những giá trị văn hóa, lịch sử đặc sắc của miền Tây Nam Bộ. Đây là nơi lưu giữ và tái hiện những nét đặc trưng văn hóa nghệ thuật dân tộc, cùng với đó là hệ sinh thái đa dạng với nhiều nhóm động vật quý hiếm. Một trong những điểm đặc biệt khi đến với khu du lịch sinh thái Tầm Vu chính là cơ hội được tìm hiểu về chiến thắng lịch sử Tầm Vu, một trang sử hào hùng của nhân dân ta trong thời kỳ kháng chiến chống Pháp.
\end{itemize}

\begin{itemize}
    \item{\textbf{Các thông tin khác - Cách di chuyển}}
\end{itemize}

\section{Đền thờ Bác Hồ}
\begin{itemize}
    \item{\textbf{Địa chỉ}} Ấp 3, Lương Tâm, Long Mỹ
    \item{\textbf{Giờ mở cửa:}} 8:00 – 16:00
    \item{\textbf{Giá vé/ chi phí:}} Miễn phí
    \item{\textbf{Phân loại điểm du lịch} } Đền thờ
    \item{\textbf{Nguồn thông tin}}
    \item{\textbf{Thông tin về địa điểm:}} Đền thờ Bác Hồ tại Hậu Giang là một địa điểm tham quan mang ý nghĩa lịch sử, nơi thể hiện lòng kính trọng và biết ơn đối với vị lãnh tụ vĩ đại của dân tộc - Chủ tịch Hồ Chí Minh. Đền thờ nằm cách thị trấn Long Mỹ 21km và cách thành phố Cần Thơ 78km về phía Tây Nam, là nơi thu hút nhiều du khách và bà con nhân dân đến viếng thăm, thắp nén hương tưởng nhớ Bác Hồ.
\end{itemize}

\begin{itemize}
    \item{\textbf{Các thông tin khác - Cách di chuyển}}
\end{itemize}


\newpage
\section*{{Ẩm thực}}
\setcounter{section}{0}

\section{Khô cá chạch}
\begin{itemize}
    \item{\textbf{Mô tả món ăn}} Khô cá chạch có thể chiến biến thành nhiều món ăn, đặc sắc nhất phải kể đến: canh chua khô cá chạch, cá chạch rim thơm, xào sả ớt,… Đây sẽ là món quà rất lý tưởng để dành tặng cho người thân và bạn bè sau chuyến du lịch này.
    \item{\textbf{Địa chỉ tham khảo}} Khô Cá Chạch, phường III, Thành phố Vị Thanh, Hậu Giang.
    \item{\textbf{Giá tham khảo:}} 200.000 – 300.000VNĐ
\end{itemize}

\begin{itemize}
    \item{\textbf{Các thông tin khác}}
\end{itemize}

\section{Cam sành Ngã Bảy}
\begin{itemize}
    \item{\textbf{Mô tả món ăn}} Cam sành Ngã Bảy vừa to, tròn, vừa mọng nước, múi cam cũng có nhiều vitamin, rất thích hợp cho những ngày nắng nóng, làm một ly cam tươi mát sẽ giải tỏa cơn nóng nực cho du khách khi đến đây.
    \item{\textbf{Địa chỉ tham khảo}}
          \begin{itemize}
              \item Ấp Bảy Thưa, xã Tân Thành, thị xã Ngã Bảy, Hậu Giang.
              \item Ấp Đông Bình, xã Tân Thành, thị xã Ngã Bảy, Hậu Giang.
          \end{itemize}
    \item{\textbf{Giá tham khảo:}} 30.000 – 50.000VNĐ
\end{itemize}

\begin{itemize}
    \item{\textbf{Các thông tin khác}}
\end{itemize}

\section{Sữa chua dê sấy khô}
\begin{itemize}
    \item{\textbf{Mô tả món ăn}} Đây là sản phẩm được nhiều khách du lịch đến tham quan và mua về làm quà với hương vị thơm ngon thu hút được lượng lớn du khách tin dùng.
    \item{\textbf{Địa chỉ tham khảo}} ấp 2B đường Nguyễn Công Trứ, xã Tân Hòa, huyện Châu Thành A, tỉnh Hậu Giang.
    \item{\textbf{Giá tham khảo:}} 60.000 – 80.000VNĐ/sản phẩm
\end{itemize}

\begin{itemize}
    \item{\textbf{Các thông tin khác}}
\end{itemize}

\section{Chả cá thác lác}
\begin{itemize}
    \item{\textbf{Mô tả món ăn}} Chả được làm từ cá thác lác cườm thơm ngon, tươi mới, được xay cùng tiêu đen hoặc thì là nên thịt chả không bị tanh. Đây sẽ là món ăn rất thích hợp trong bữa cơm gia đình.
    \item{\textbf{Địa chỉ tham khảo}}
          \begin{itemize}
              \item Nhà hàng Tân Hậu Giang – số 33 Ba Tháng Hai, Phường 5, Vị Thanh, Hậu Giang.
              \item Cá thác lác Kim Ngoan Hậu Giang – số 1/2/6, ấp 2, xã Vị Đông, huyện Vị Thuỷ, tỉnh Hậu Giang.
          \end{itemize}
    \item{\textbf{Giá tham khảo:}} 250.000 – 300.000VNĐ
\end{itemize}

\begin{itemize}
    \item{\textbf{Các thông tin khác}}
\end{itemize}

\section{Sỏi mầm}
\begin{itemize}
    \item{\textbf{Mô tả món ăn}} Sỏi mầm là đặc sản nổi tiếng ở Hậu Giang. Thay vì nướng trên vỉ bếp du khách có thể nướng trực tiếp trên sỏi đã nóng, ăn miếng nào thực khách áp lên đá cho đến khi thịt chín vàng đều là ăn được. Thịt ăn chung với rau sống, chấm nước mắm chua ngọt thu hút thực khách. Không khói than, không lửa bếp nhưng lại có món thịt nướng thơm thơm. Đây sẽ là một trải nghiệm thú vị, du khách nên thử qua khi ghé đến Hậu Giang.
    \item{\textbf{Các thông tin khác}}
\end{itemize}

\section{Cháo lòng Cái Tắc}
\begin{itemize}
    \item{\textbf{Mô tả món ăn}} Cháo lòng thì chỗ nào cũng có nhưng điểm đặc biệt để làm nên thương hiệu là ở chỗ cách nêm nếm, cháo được nấu nóng hổi có thêm lòng lợn, gia vị và ăn kèm rau nhưng lại khiến thực khách không khỏi xuýt xoa với món ăn ngon miệng này.
    \item{\textbf{Các thông tin khác}}
\end{itemize}

\section{Cá ngát}
\begin{itemize}
    \item{\textbf{Mô tả món ăn}} Cá ngát sau khi qua bàn tay khéo léo của người miền Tây được chế biến thành những món ăn vô cùng hấp dẫn. Trong đó, cá ngát kho tộ là món ăn thường trực trong bữa cơm của người dân sông nước miền Tây. Cá làm sạch sẽ được tẩm ướp gia vị thêm chút tiêu, ớt rồi cho lên bếp đun lửa riu riu. Lẩu cá ngát chua chua, thịt cá ngát ngọt thanh, chắc chắn sẽ mang đến cho du khách một món ăn không thể nào tuyệt vời hơn vào những ngày mưa.
    \item{\textbf{Các thông tin khác}}
\end{itemize}

\section{Lẩu mẻ}
\begin{itemize}
    \item{\textbf{Mô tả món ăn}} Lẩu mẻ Hậu Giang được ăn cùng với thịt gà, bò, cá,… Nước lẩu được nấu từ nước dừa xiêm để có vị ngọt thanh, thêm mẻ rồi nêm gia vị vừa ăn. Lẩu ăn kèm với bắp chuối bào sợi, bạc hà, rau muống,… Vị ngọt thanh của dừa, vị chua tê tái của mẻ, vị tươi ngọt của cá đọng lại. Lẩu mẻ Hậu Giang sẽ là một trong những món khiến thực khách phải nhớ mãi.
    \item{\textbf{Các thông tin khác}}
\end{itemize}

\section{Gà hầm sả}
\begin{itemize}
    \item{\textbf{Mô tả món ăn}} Hương thơm của sả, một chút ngọt thịt của gà giúp cho nước dùng trở nên đậm vị hơn bao giờ hết. Gà hầm cùng sả có thể cho thêm củ cải trắng, cà rốt, nấm để dậy lên vị của thịt gà. Thịt gà dai dai mềm mềm, nước gà thanh thanh, thơm thơm mùi sả thật khiến cho người ăn phải đắm chìm vào hương vị ngất ngây này.
    \item{\textbf{Các thông tin khác}}
\end{itemize}

\section{Chả cá thác lác}
\begin{itemize}
    \item{\textbf{Mô tả món ăn}} Cá thác lác là một loại cá nước ngọt, sống nhiều ở các kênh, rạch. Chả cá thác lác có thể chế biến thành nhiều món ăn như canh chua chả cá ăn cùng bông điên điển hay bông súng, chả cá chiên ăn cùng tương ớt,… nghe thôi cũng đủ để khiến thực khách phải nao lòng.
    \item{\textbf{Các thông tin khác}}
\end{itemize}

\section{Khóm Cầu Đúc Hậu Giang}
\begin{itemize}
    \item{\textbf{Mô tả món ăn}} Khóm Cầu Đúc nổi tiếng to và ngọt đặc biệt là rất thơm, khóm để lâu và bảo quản trong ngăn lạnh vẫn được. Khóm dùng để nấu ăn hay ép nước đều rất ngon, không những thế, người miền Tây còn chế biến khóm thành nhiều món ngon. Trái khóm còn được chế biến thành các đặc sản làm quà khác như mức khóm, kẹo khóm, rượu khóm, khóm phơi khô,…
    \item{\textbf{Địa chỉ tham khảo}} Làng du lịch cộng đồng Cánh Đồng Khóm Cầu Đúc (Hậu Giang) – ấp Thạnh Thắng, xã Hoả Tiến, thành phố Vị Thanh, Hậu Giang.
    \item{\textbf{Giá tham khảo:}} Miễn phí
    \item{\textbf{Giờ mở cửa:}} 24/24
\end{itemize}

\section{Bưởi Năm Roi Phú Hữu}
\begin{itemize}
    \item{\textbf{Mô tả món ăn}} Bưởi năm roi Phú Hữu là loại bưởi được đánh giá cao về chất lượng cả số lượng. Vỏ bưởi căng đều và có màu vàng ngà, tép bưởi to mộng nước. Đến đây, du khách hãy mang đặc sản của Phú Hữu về làm quà cho bạn bè và gia đình dùng thử xem sao nhé, sẽ rất tuyệt vời đấy.
    \item{\textbf{Địa chỉ tham khảo}} ấp Phú Trí – Phú Hữu – Châu Thành – Hậu Giang.
    \item{\textbf{Giá tham khảo:}} 30.000 – 60.000VNĐ
\end{itemize}


\end{document}