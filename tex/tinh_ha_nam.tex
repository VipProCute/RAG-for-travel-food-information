\documentclass{article}
\usepackage[vietnamese]{babel}
\usepackage[letterpaper,top=1cm,bottom=1cm,left=1.5cm,right=1.5cm,marginparwidth=1.75cm]{geometry}
\usepackage{amsmath}
\usepackage{graphicx}
\usepackage[colorlinks=true, allcolors=blue]{hyperref}
\title{Cẩm nang du lịch Tỉnh Hà Nam}

\begin{document}
\begin{center}
    \fontsize{18}{20}\textbf{Cẩm nang du lịch Tỉnh Hà Nam}
\end{center}
\begin{abstract}
    (Giới thiệu)
\end{abstract}
\section*{Điểm du lịch}

\section{Nhà Bá Kiến}
\begin{itemize}
    \item{\textbf{Địa chỉ}} Hoà Hậu, Lý Nhân, Hà Nam

    \item{\textbf{Giờ mở cửa:}} Cả ngày

    \item{\textbf{Giá vé/ chi phí:}} Miễn phí

    \item{\textbf{Phân loại điểm du lịch}} 

    \item{\textbf{Nguồn thông tin}} 

    \item{\textbf{Thông tin về địa điểm:}} Hiện tại nhà ba gian của Bá Kiến trong nguyên mẫu truyện Chí Phèo vẫn còn là một địa điểm du lịch Hà Nam thu hút khách tham quan. Nguyên Mẫu làng Vũ Đại chính là làng Đại Hoàng, thôn Nhân Hậu, xã Hòa Hậu, huyện Lý Nhân, tỉnh Hà Nam - quê hương của nhà văn Nam Cao. Nhà vốn do ngụy viên Bắc kỳ Bá Bính sở hữu, tên thật là Trần Duy Bính và cũng chính là nguyên mẫu của nhân vật Bá Kiến trong câu chuyện.
\end{itemize}

\begin{itemize}
    \item{\textbf{Các thông tin khác - Cách di chuyển}} 
\end{itemize}

\section{Làng kho cá Vũ Đại}
\begin{itemize}
    \item{\textbf{Địa chỉ}} Làng Nhân Hậu, xã Hòa Hậu, huyện Lý Nhân, Hà Nam

    \item{\textbf{Giờ mở cửa:}} Cả ngày

    \item{\textbf{Giá vé/ chi phí:}} Miễn phí

    \item{\textbf{Phân loại điểm du lịch}} 

    \item{\textbf{Nguồn thông tin}} 

    \item{\textbf{Thông tin về địa điểm:}} Một địa điểm du lịch Hà Nam vẫn luôn nổi tiếng với công thức cá kho đặc biệt, đậm đà và được tẩm ướp kỹ lưỡng với đủ loại gia vị đặc trưng như: riềng, gừng, nước cốt chanh, ớt...mách bạn có thể đến đây mua cá kho trong nồi đất. Đặc biệt cá kho bán vô cùng đắt hàng trong những dịp cận Tết, nhiều hộ dân thậm chí còn phải kho cá ngày đêm mới kịp số lượng bán ra.
\end{itemize}

\begin{itemize}
    \item{\textbf{Các thông tin khác - Cách di chuyển}} 
\end{itemize}

\section{Đền Trần Thương}
\begin{itemize}
    \item{\textbf{Địa chỉ}} Thôn Trần Thương, xã Nhân Đạo, huyện Lý Nhân, Hà Nam

    \item{\textbf{Giờ mở cửa:}} Cả ngày

    \item{\textbf{Giá vé/ chi phí:}} Miễn phí

    \item{\textbf{Phân loại điểm du lịch}} 

    \item{\textbf{Nguồn thông tin}} 

    \item{\textbf{Thông tin về địa điểm:}} Đền Trần Thương là địa điểm du lịch Hà Nam nằm ngay bên bờ sông Hồng, thờ Trần Hưng Đạo. Đền được dựng lên trên phần đất từng được đại vương dựng làm kho lương trong kháng chiến chống quân Nguyên – Mông lần thứ hai. Kinh nghiệm du lịch mách bạn nên ghé thăm đền vào tháng Giêng, đây là thời điểm thường tổ chức lễ phát lương lấy may đầu xuân và tưởng nhớ công hơn của Hưng Đạo Vương Trần Quốc Tuấn.
\end{itemize}

\begin{itemize}
    \item{\textbf{Các thông tin khác - Cách di chuyển}} 
\end{itemize}

\section{Chùa Tam Chúc}
\begin{itemize}
    \item{\textbf{Địa chỉ}} Thị trấn Ba Sao, huyện Kim Bảng, Hà Nam

    \item{\textbf{Giờ mở cửa:}} Cả ngày

    \item{\textbf{Giá vé/ chi phí:}} Miễn phí

    \item{\textbf{Phân loại điểm du lịch}} 

    \item{\textbf{Nguồn thông tin}} 

    \item{\textbf{Thông tin về địa điểm:}} Đi du lịch chùa Tam Chúc bạn sẽ được chiêm ngưỡng khung cảnh ngôi chùa lớn nhất Việt Nam, chỉ cách trung tâm Hà Nội khoảng 60km và thu hút đông đảo du khách ghé thăm mỗi năm. Chùa là điểm kết nối giữa chùa Hương, Bái Đính và khu bảo tồn thiên nhiên Vân Long. Một số điểm tham quan nổi bật tại đây có thể kể đến như vườn cột kinh, Điện Tam Thế, Điện Quán Thế Âm Bồ Tát, Chùa Ngọc trên đỉnh núi Thất Tinh...
\end{itemize}

\begin{itemize}
    \item{\textbf{Các thông tin khác - Cách di chuyển}} 
\end{itemize}

\section{Đền Lảnh Giang}
\begin{itemize}
    \item{\textbf{Địa chỉ}} Làng Yên Lạc, Duy Tiên, Hà Nam

    \item{\textbf{Giờ mở cửa:}} Cả ngày

    \item{\textbf{Giá vé/ chi phí:}} Miễn phí

    \item{\textbf{Phân loại điểm du lịch}} 

    \item{\textbf{Nguồn thông tin}} 

    \item{\textbf{Thông tin về địa điểm:}} Đền Lảnh Giang thờ Tam vị danh thần từ đời Hùng Vương thứ 18 và tọa lạc trong một khuôn viên rộng đến 3.000m2. Không khí núi đồi xanh mát cùng rừng cây trái, bên nước, đầm sen như phảng phất màu sắc của một mảnh đất địa linh nhân kiệt, phồn thịnh và êm đềm. Đến với đền Lảnh Giang bạn sẽ được chiêm bái Tam vị danh thần từ đời Hùng Vương thứ 18.
\end{itemize}

\begin{itemize}
    \item{\textbf{Các thông tin khác - Cách di chuyển}} 
\end{itemize}

\section{Đền Trúc}
\begin{itemize}
    \item{\textbf{Địa chỉ}} Quyển Sơn, Kim Bảng, Hà Nam

    \item{\textbf{Giờ mở cửa:}} Cả ngày

    \item{\textbf{Giá vé/ chi phí:}} Miễn phí

    \item{\textbf{Phân loại điểm du lịch}} 

    \item{\textbf{Nguồn thông tin}} 

    \item{\textbf{Thông tin về địa điểm:}} Ngôi đền độc đáo được dựng bằng gỗ lim và rợp bóng trúc bao xung quanh. Sắc xanh hài hòa cùng mái ngói cổ kính đặc biệt được khách du lịch Hà Nam yêu thích. Đặc biệt nơi đây còn có lễ hội hát dặm ý nghĩa để tôn vinh anh hùng Lý Thường Kiệt của dân tộc.
\end{itemize}

\begin{itemize}
    \item{\textbf{Các thông tin khác - Cách di chuyển}} 
\end{itemize}

\section{Chùa Địa Tạng Phi Lai Tự}
\begin{itemize}
    \item{\textbf{Địa chỉ}} Thôn Ninh Trung, xã Liêm Sơn, Thanh Liêm, Hà Nam

    \item{\textbf{Giờ mở cửa:}} Cả ngày

    \item{\textbf{Giá vé/ chi phí:}} Miễn phí

    \item{\textbf{Phân loại điểm du lịch}} 

    \item{\textbf{Nguồn thông tin}} 

    \item{\textbf{Thông tin về địa điểm:}} Địa điểm du lịch Hà Nam này dù mới xây dựng cách đây không lâu nhưng được đông đảo khách hành hương yêu thích. Chùa Địa Tạng Phi Lai Tự nằm tựa lưng vào núi, hai bên là dãy núi theo thế "tả thanh long, hữu bạch hổ" với nhiều cổ vật thiêng liêng. Bên cạnh đó chùa còn được rừng thông cao vút, yên bình bao bọc xung quanh.
\end{itemize}

\begin{itemize}
    \item{\textbf{Các thông tin khác - Cách di chuyển}} 
\end{itemize}

\section{Chùa Bà Đanh}
\begin{itemize}
    \item{\textbf{Địa chỉ}} Thôn Đanh, Ngọc Sơn, Kim Bảng, Hà Nam

    \item{\textbf{Giờ mở cửa:}} Cả ngày

    \item{\textbf{Giá vé/ chi phí:}} Miễn phí

    \item{\textbf{Phân loại điểm du lịch}} 

    \item{\textbf{Nguồn thông tin}} 

    \item{\textbf{Thông tin về địa điểm:}} Chùa Bà Đanh sở hữu cảnh quan hữu tình mà cô tịch, có diện tích khoảng 10ha, được xem là một trong những ngôi chùa đẹp và cổ kính nhất tại Hà Nam nói riêng và của các tỉnh miền Bắc nói chung. Khuôn viên bên trong chùa bao gồm gần 40 gian nhà lớn nhỏ đậm nét kiến trúc nghệ thuật. Khách tham quan đến đây nhất định không được bỏ lỡ cơ hội khám phá lễ hội Chùa Bà Đanh được tổ chức vào tháng 2 âm lịch hàng năm.
\end{itemize}

\begin{itemize}
    \item{\textbf{Các thông tin khác - Cách di chuyển}} 
\end{itemize}


\newpage
\section*{{Ẩm thực}}
\setcounter{section}{0}

\end{document}